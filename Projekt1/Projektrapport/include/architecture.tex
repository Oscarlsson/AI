\chapter{Architecture}
Describe the different parts of your program suite in detail.
\\\\
kort inledning till kapitlet. vad har vi gjort i projektet?\\
trattigt namn, borde kanske ändras. ska innehålla typ metod - alltså vad har vi gjort?

\section{Finished work}
Running modules
\\\\
beskriv vilka delar som funkar och är klara t.ex.\\
* naive bayes\\
* perceptron\\
* average perceptron\\
* knn\\
* svm
\\\\
What does your running code do? what is the output?

\subsection{The Perceptron}
The perceptron algorithm works as follows: 
The output, w, from the algorithm is a weight vector with the same size as y. If the classification for a x in the training set is equal to y, the weight is unchanged.
If y is 1 and the classification 0, then w is increased. If y is 0 and the classification 1, then w is decreased. [reference]
\begin{verbatim}
Let N = number of iterations through training set
Initialize the weight vector w to all zeros

Iterate N times through the training set:
    For each x, y in the training set:
        Let guess = classify(x) using our current w
        If y is not equal to guess:
            w := w + f(x, y) - f(x, guess)
Return w
\end{verbatim}
The averaged perceptron works in the same way except that it returns the average weight vector instead of the final.
This averaged vector is built incrementally by updating it while we build the usual weight vector. The pseudo code is the same as for the usual perceptron, except some lines which are shown below.
\begin{verbatim}
Initialize the average weight vector wa to all zeros and the counter c to 0.

If y is not equal to guess:
    w := w + f(x, y) - f(x, guess)
    average_weight := (N*T - c) / (N*T)
    wa := wa + average_weight * (f(x, y) - f(x, guess))
Increase the counter c
\end{verbatim}


