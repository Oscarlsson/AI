\usepackage[utf8]{inputenc}
\usepackage[T1]{fontenc}
\usepackage[english]{babel}
\usepackage{amsmath}
\usepackage{amsfonts}
\usepackage{ae}
\usepackage{icomma}
\usepackage{units}
\usepackage{color}
\usepackage{graphicx}
\usepackage{epstopdf}
\usepackage{subfigure}
\usepackage{bbm}
\usepackage{caption}
\usepackage[numbers, sort]{natbib}
\usepackage{multirow}
\usepackage{array}
\usepackage{geometry}
\usepackage{fancyhdr}
\usepackage{fncychap}
\usepackage[hyphens]{url}
\usepackage[breaklinks,pdfpagelabels=false]{hyperref}
\usepackage{lettrine}
\usepackage{eso-pic}
\usepackage{longtable}
\usepackage{color}
\usepackage{listings}
\usepackage{rotating}
\usepackage{wrapfig}
% http://ctan.uib.no/macros/latex/contrib/algorithm2e/doc/algorithm2e.pdf
\usepackage[algoruled]{algorithm2e}

\lstset{ %
language=SQL,                % choose the language of the code
%basicstyle=\footnotesize,       % the size of the fonts that are used for the code
%numbers=left,                   % where to put the line-numbers
%numberstyle=\footnotesize,      % the size of the fonts that are used for the line-numbers
%stepnumber=1,                   % the step between two line-numbers. If it is 1 each line will be numbered
%numbersep=5pt,                  % how far the line-numbers are from the code
backgroundcolor=\color{white},  % choose the background color. You must add \usepackage{color}
showspaces=false,               % show spaces adding particular underscores
showstringspaces=false,         % underline spaces within strings
showtabs=false,                 % show tabs within strings adding particular underscores
%frame=single,           % adds a frame around the code
tabsize=2,          % sets default tabsize to 2 spaces
captionpos=b,           % sets the caption-position to bottom
breaklines=true,        % sets automatic line breaking
breakatwhitespace=false,    % sets if automatic breaks should only happen at whitespace
%escapeinside={\%*}{*)}          % if you want to add a comment within your code
}

\newcommand{\rd}{\ensuremath{\mathrm{d}}}
\newcommand{\id}{\ensuremath{\,\rd}}
\newcommand{\degC}{\ensuremath{\,\unit{^\circ C}}}

% Fancyheader shortcuts
\newcommand{\setdefaulthdr}{%
\fancyhead[L]{\slshape \rightmark}%
\fancyhead[R]{\slshape \leftmark}%
\fancyfoot[C]{\thepage}%
}
\newcommand{\setspecialhdr}{%
\fancyhead[L]{ }%
\fancyhead[R]{\slshape \leftmark}%
\fancyfoot[C]{\thepage}%
}


\newcommand{\mail}[1]{\href{mailto:#1}{\nolinkurl{#1}}}
\newcommand{\backgroundpic}[3]{%
	\put(#1,#2){
		\parbox[b][\paperheight]{\paperwidth}{%
			\centering
			\includegraphics[width=\paperwidth,height=\paperheight,keepaspectratio]{#3}
			\vfill
}}}

\DeclareMathOperator*{\argmax}{arg\,max}
\DeclareMathOperator*{\argmin}{arg\,min}