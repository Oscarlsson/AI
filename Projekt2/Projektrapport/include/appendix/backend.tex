\section{Backend}
\begin{center}
\line(1,0){418}
\end{center}
\begin{verbatim}
module Backend (
    Ground, Indexes, World(W, holding, ground, indexes, wsize),
    findBlockByName, isLeftOf, isRightOf, isAbove, isUnder, isOnGround,
    isOnTop, isOnTop’, isOnBottom, isBeside, isHolding, isEmptyIndex,
    isOnPoss, createWorld, createGround, createIndexes, createBlocks,
    createBlocks’, getBlock, getBlockIndex, getBlocksAt,
    getBlocksOnGroundBy, getRightMost, getLeftMost, getUnderMost,
    getUpperMost, getUpperUnderMost, getMinimumStackHeight,
    getMinimumStackHeightFrom, getMinimumStackHeightUntil, initWorld
) where
\end{verbatim}
\begin{center}
\line(1,0){418}
\end{center}
\begin{verbatim}
type Ground = Map Int [Block]
\end{verbatim}
\begin{tabbing}
\hspace*{1cm}\= \kill
\> The key is the x-coordinate and the value is the blocks at this index
\end{tabbing}
\begin{verbatim}
type Indexes = Map Block Int
\end{verbatim}
\begin{tabbing}
\hspace*{1cm}\= \kill
\> The key is a block in the world and the value the x-coordinate of this block
\end{tabbing}
\begin{verbatim}
data World = W 
        holding :: Maybe Block
        ground :: Ground
        indexes :: Indexes
        wsize :: Int

instance Eq World
instance Ord World
instance Show World
\end{verbatim}
\begin{verbatim}
findBlockByName :: String -> World -> Maybe Block
\end{verbatim}
\begin{tabbing}
\hspace*{1cm}\= \kill
\> Takes the name of a block and returns this block if it exist in the world
\end{tabbing}
\begin{verbatim}
isLeftOf :: Block -> Block -> World -> Bool
\end{verbatim}
\begin{tabbing}
\hspace*{1cm}\= \kill
\> Check if the first input block is to the left of the second input block
\end{tabbing}
\begin{verbatim}
isRightOf :: Block -> Block -> World -> Bool
\end{verbatim}
\begin{tabbing}
\hspace*{1cm}\= \kill
\> Check if the first input block is to the right of the second input block
\end{tabbing}
\begin{verbatim}
isAbove :: Block -> Block -> World -> Bool
\end{verbatim}
\begin{tabbing}
\hspace*{1cm}\= \kill
\> Check if the first input block is above the second input block
\end{tabbing}
\begin{verbatim}
isUnder :: Block -> Block -> World -> Bool
\end{verbatim}
\begin{tabbing}
\hspace*{1cm}\= \kill
\> Check if the first input block is under the second input block
\end{tabbing}
\begin{verbatim}
isOnGround :: Block -> World -> Bool
\end{verbatim}
\begin{tabbing}
\hspace*{1cm}\= \kill
\> Check if a block is on the ground in a world or equally the block exists in the \\
\> world and is not the current holding
\end{tabbing}
\begin{verbatim}
isOnTop :: Block -> Block -> World -> Bool
\end{verbatim}
\begin{tabbing}
\hspace*{1cm}\= \kill
\> Check if a block is on the top of any stack in a world
\end{tabbing}
\begin{verbatim}
isOnTop’ :: Block -> Block -> World -> Bool
\end{verbatim}
\begin{tabbing}
\hspace*{1cm}\= \kill
\> Check if the first input block lies directly over the second input block
\end{tabbing}
\begin{verbatim}
isOnBottom :: Block -> World -> Bool
\end{verbatim}
\begin{tabbing}
\hspace*{1cm}\= \kill
\> Check if a block is at the bottom of any stack in a world
\end{tabbing}
\begin{verbatim}
isBeside :: Block -> Block -> World -> Bool
\end{verbatim}
\begin{tabbing}
\hspace*{1cm}\= \kill
\> Check if the first input block is either left or right of second input block in a \\
\> given world
\end{tabbing}
\begin{verbatim}
isHolding :: Block -> World -> Bool
\end{verbatim}
\begin{tabbing}
\hspace*{1cm}\= \kill
\> Check if a block is a holding in the world
\end{tabbing}
\begin{verbatim}
isEmptyIndex :: Int -> World -> Bool
\end{verbatim}
\begin{tabbing}
\hspace*{1cm}\= \kill
\> Check if a there exist blocks on a given x-coordinate in world
\end{tabbing}
\begin{verbatim}
isOnPoss :: Block -> Int -> World -> Bool
\end{verbatim}
\begin{tabbing}
\hspace*{1cm}\= \kill
\> Check if a x-coordinate holds the input block
\end{tabbing}
\begin{verbatim}
createWorld :: [[String]] -> String -> [Block] -> Maybe World
\end{verbatim}
\begin{tabbing}
\hspace*{1cm}\= \kill
\> 
\end{tabbing}
\begin{verbatim}
createGround :: [[String]] -> [Block] -> Maybe Ground
\end{verbatim}
\begin{tabbing}
\hspace*{1cm}\= \kill
\> Used by createWorld
\end{tabbing}
\begin{verbatim}
createIndexes :: [[String]] -> [Block] -> Maybe Indexes
\end{verbatim}
\begin{tabbing}
\hspace*{1cm}\= \kill
\> Used by createWorld
\end{tabbing}
\begin{verbatim}
createBlocks :: Int -> [String] -> [Block] -> Maybe [(Block, Int)]
\end{verbatim}
\begin{tabbing}
\hspace*{1cm}\= \kill
\> Used by createWorld
\end{tabbing}
\begin{verbatim}
createBlocks’ :: Int -> [String] -> [Block] -> Maybe (Int, [Block])
\end{verbatim}
\begin{tabbing}
\hspace*{1cm}\= \kill
\> Used by createWorld
\end{tabbing}
\begin{verbatim}
getBlock :: String -> [Block] -> Maybe Block
\end{verbatim}
\begin{tabbing}
\hspace*{1cm}\= \kill
\> Input : a name of a block , any list of blocks Output: The block with the given \\
\> name from the list if it exist
\end{tabbing}
\begin{verbatim}
getBlockIndex :: Block -> World -> Maybe Int
\end{verbatim}
\begin{tabbing}
\hspace*{1cm}\= \kill
\> Input: Any block, a word return x-coordinate of the given block if it exist \\
\> in the world
\end{tabbing}
\begin{verbatim}
getBlocksAt :: Int -> World -> Maybe [Block]
\end{verbatim}
\begin{tabbing}
\hspace*{1cm}\= \kill
\> Input: A x-coordinate, a world Output: All blocks at the given index if the \\
\> index exists in the world
\end{tabbing}
\begin{verbatim}
getBlocksOnGroundBy :: (Block -> Bool) -> World -> [Block]
\end{verbatim}
\begin{tabbing}
\hspace*{1cm}\= \kill
\> Input: a function, a world Output Returns all blocks an the ground that \\
\> satifies the given function
\end{tabbing}
\begin{verbatim}
getRightMost :: [Block] -> World -> Maybe Block
\end{verbatim}
\begin{tabbing}
\hspace*{1cm}\= \kill
\> Input: Any list a blocks, a world Output: The first occurance of a block with \\
\> the biggest x-coordinate if all given blocks exist on the ground in the given list
\end{tabbing}
\begin{verbatim}
getLeftMost :: [Block] -> World -> Maybe Block
\end{verbatim}
\begin{tabbing}
\hspace*{1cm}\= \kill
\> The same as rightmost but with smallest index
\end{tabbing}
\begin{verbatim}
getUnderMost :: [Block] -> World -> Maybe Block
\end{verbatim}
\begin{tabbing}
\hspace*{1cm}\= \kill
\> Input: any list of blocks, a world Output: If the input blocks havn’t the same \\
\> index nothing is returned else the bottom block from the input list in the stack
\end{tabbing}
\begin{verbatim}
getUpperMost :: [Block] -> World -> Maybe Block
\end{verbatim}
\begin{tabbing}
\hspace*{1cm}\= \kill
\> 
\end{tabbing}
\begin{verbatim}
getUpperUnderMost :: ([Block] -> Block) -> [Block] -> World -> Maybe Block
\end{verbatim}
\begin{tabbing}
\hspace*{1cm}\= \kill
\> Used by getUnderMost and getUpperMost
\end{tabbing}
\begin{verbatim}
getMinimumStackHeight :: World -> Int
\end{verbatim}
\begin{tabbing}
\hspace*{1cm}\= \kill
\> Returns the minimum stack height in the world
\end{tabbing}
\begin{verbatim}
getMinimumStackHeightFrom :: World -> Int -> Maybe Int
\end{verbatim}
\begin{tabbing}
\hspace*{1cm}\= \kill
\> Works like getMinimumStackHeight but uses a start index to look from\\
\> getMinimumStackHeight w == getMinimumStackHeightFrom w 0
\end{tabbing}
\begin{verbatim}
getMinimumStackHeightUntil :: World -> Int -> Maybe Int
\end{verbatim}
\begin{tabbing}
\hspace*{1cm}\= \kill
\> Works like getMinimumStackHeight but uses a last index to look at \\ \> getMinimumStackHeight w == getMinimumStackHeightUntil w (worldsize - 1)
\end{tabbing}
\begin{verbatim}
initWorld :: [[[Char]]]
\end{verbatim}
\begin{tabbing}
\hspace*{1cm}\= \kill
\> For testing purposes
\end{tabbing}
