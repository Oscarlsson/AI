\chapter{Architecture}
The parser and the planner was implemented in Haskell. The parser interpreted what blocks the sentence refered to and the planner found a planning of how to move these blocks to reach the goal state determined by the parser. 

\section{Programs and tools}
Haskell was used to implement the parser and the planner. The given grammar was written in GF and the interface used CGI and JavaScript. A Python script is used to handle the communication between the Haskell files and the JavaScripts.
\\\\
The planner was implemented using a standard library priority queue \citep{psq_url}. 

\section{Backend structure}
The parser and the planner shared a number of common functions and structures, which were defined in two separate modules found in Appendix B.1 and Appendix B.2. These modules defined how to create a representation of a world from the given sentence and functions such as to check whether a certain block is relative to another block or not. 

\section{Parser}
The sentence was first modified by removing unnessesary characters, such as ''.'', ''!'' and ''?''. Also, all characters were changed to lower case so it could be given and parsed by the GF grammar. To be able to use pattern matching on the generated output from the GF parser, the GF grammar was translated into Haskell data types, which were automatically generated (See Appendix B.7). From this grammar an abstract syntax was generated for the input to be used for pattern matching. If the input was ambiguous, several parse trees were generated and passed to the planner. \\\\
For each parse tree, the tree was translated to an output which was given to the planner. Since some parse trees could refer to, for example, non existing objects in the world the parser used an Error Monad to tell if the given tree was OK or not. \\\\
The output to the planner (fully stated in Appendix B.5) consists of three parts:
\begin{itemize}
\item An action to do ($Move$, $Put$ or $Take$)
\item A list of blocks to be moved
\item A location where to move the blocks, which consists of an reference ($Above$, $Beside$, $Inside$, $LeftOf$, $OnTop$, $RightOf$ or $Under$) and a list of blocks which the reference refers to or $Floor$ and a list of stack indexes where the stack is empty.
\end{itemize}
The output differed depending on which action to do. E.g. the action $Take$ did not have a location and the action $Put$ would only have the block in holding as the list of blocks to be moved. If holding is $null$, $Put$ would do nothing since it assumes that there is a block in holding in contrast to the action $Move$.\\\\
The parser would always find out if the blocks the input refered to exists and if not, give an error. All laws of physics except where the robot is allowed to drop blocks is preserved within the parser. It would be impossible to take a block left of or right of all blocks, since this refers to the block itself. And it would not be allowed to take the floor or put something inside a block that is not a box. All blocks must also be above the floor. If any of these laws were violated, an error were returned.
 \\\\
If a block should be put left of or right of several blocks, the parser returns the leftmost or rightmost block as a reference block. 

\section{Planner}
* goal, finished \\
* heuristic function\\
* A*, nodes and edges, timeout\\
* instructions, valid, comparator\\
* pseudo code A*\\\\
Successors på en värld returnerar alla möjliga världar man kan komma till genom att göra en tillåten instruktion. Add räknar ut f(n) på varje värld
\begin{algorithm}[h!]
 \SetAlgoLined
 \KwData{TimeoutInt, World, Goal}
 \KwResult{History}
 Initialize priority queue $PQ$ with world, empty history and value 0\;
 $Seen \leftarrow empty$\;
 \While{not timeout}{
   $World$, $History$ $\leftarrow$ pop $PQ$\;
   \If {$World == Goal$}{
    break\;
   }
   $Seen \leftarrow World + Seen$\; 
   $Succ \leftarrow$ successors of $World$ not in $Seen$\;
   $PQ' \leftarrow$ add $Succ$\;
 }
 \Return $History$
 \caption{A*}
 \label{algorithm:astar}
\end{algorithm}
