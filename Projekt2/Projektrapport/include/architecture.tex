\chapter{Architecture}
The parser and the planner were implemented in Haskell. The parser interpreted
what blocks the sentence referred to and the planner found a plan of how to
move these blocks to reach the goal state determined by the parser. 

\section{Programs and tools}
Haskell was used to implement the parser and the planner. The given grammar was
written in GF and the interface used CGI and JavaScript. A Python script is
used to handle the communication between the Haskell files and the JavaScripts.
\\\\
The planner was implemented using a standard library priority queue \citep{psq_url}. 

\section{Backend structure}
The parser and the planner shared a number of common functions and structures,
which were defined in two separate modules found in Appendix B.1 and Appendix
B.2. These modules defined how to create a representation of a world from the
given sentence and functions such as to check whether a certain block is
relative to another block or not. 

\section{Parser}
The sentence was first modified by removing unnecessary characters, such as
''.'', ''!'' and ''?''. Also, all characters were changed to lowercase so it
could be given and parsed by the GF grammar. To be able to use pattern matching
on the generated output from the GF parser, the GF grammar was translated into
Haskell data types, which were automatically generated (See Appendix B.7). From
this grammar an abstract syntax was generated for the input to be used for
pattern matching. If the input was ambiguous, several parse trees were
generated and passed to the planner. \\\\
For each parse tree, the tree was translated to an output which was given to
the planner. Since some parse trees could refer to, for example, non existing
objects in the world the parser used an Error Monad to tell if the given tree
was OK or not. \\\\
The output to the planner (fully stated in Appendix B.5) consists of three parts:
\begin{itemize}
\item An action to do ($Move$, $Put$ or $Take$)
\item A list of blocks to be moved
\item A location where to move the blocks, which consists of an reference
        ($Above$, $Beside$, $Inside$, $LeftOf$, $OnTop$, $RightOf$ or $Under$)
        and a list of blocks which the reference refers to or $Floor$ and a
        list of stack indexes where the stack is empty.
\end{itemize}
The output differed depending on which action to do. E.g. the action $Take$ did
not have a location and the action $Put$ would only have the block in holding
as the list of blocks to be moved. If holding is $null$, $Put$ would do nothing
since it assumes that there is a block in holding in contrast to the action
$Move$.\\\\
The parser would always find out if the blocks the input referred to exists and
if not, give an error. All laws of physics except where the robot is allowed to
drop blocks is preserved within the parser. It would be impossible to take a
block left of or right of all blocks, since this refers to the block itself.
And it would not be allowed to take the floor or put something inside a block
that is not a box. All blocks must also be above the floor. If any of these
laws were violated, an error was returned.
 \\\\
If a block should be put left of or right of several blocks, the parser returns
the leftmost or rightmost block as a reference block. 

\section{Planner}
The planner uses $A^*$ similar to Algorithm \ref{algorithm:astar}. The
successor function in a world returns every possible world that can be reached
by applying any valid instruction. The add function is responsible for
calculating the heuristics of every node.
\\\\
Every node in a world consist of a history which is a sequence of instructions. 
The nodes were saved in a priority queue and two equal words are never saved more
then once. If a world from the queue is popped and satisfy a finished state
then the algorithm has found a solution and terminates. This in contrast to if a
successor is found that satisfy the same constraints then optimality can not be
guaranteed. This because of that one must take the heuristic in consideration. \\ 
\begin{algorithm}[h!]
 \SetAlgoLined
 \KwData{TimeoutInt, World, Goal}
 \KwResult{History}
 Initialize priority queue $PQ$ with world, empty history and value 0\;
 $Seen \leftarrow empty$\;
 \While{not timeout}{
   $World$, $History$ $\leftarrow$ pop $PQ$\;
   \If {$World == Goal$}{
    break\;
   }
   $Seen \leftarrow World + Seen$\; 
   $Succ \leftarrow$ successors of $World$ not in $Seen$\;
   $PQ' \leftarrow$ add $Succ$\;
 }
 \Return $History$
 \caption{A*}
 \label{algorithm:astar}
\end{algorithm} 
\\
To make sure of that a solution is found a world must satisfy a number of logical
constraints. This is generated from an interpretation of what the user has
written. This interpretation comes from the parsing of the natural language but 
also choices from what a given parse tree actually means. This logical
constraints differ for different parse trees.   
\\\\
The implemented $A^*$ algorithm makes use of a timeout. This means that 
the algorithm terminates after a number of steps even though a solution is not
found. This generally happens when a solution is impossible to find. 
\\\\
The laws of physics in the world are handled by comparison functions. A pyramid
and a ball are always smaller than every other object. In this way it is never
possible to put anything on a ball or pyramid. 
\\\\
The heuristic assumes that that there are infinitely many free spaces in the
world where one can but a block. From this follows that one can always move a
free block in two steps. 
%Varje nod är en värld och en historia som består av en sekvens av
%instruktioner.  Noderna sparas i en prioritetskö och två lika dana värdar
%sparas aldrig mer än en gång. 

%Om vi poppar en värld ur prioritetskön som också
%uppfyller finished så är vi färdiga till skillnad från om vi hade hittat en
%succsessor som var finishid för då kan vi inte garantera optimalitet. Detta på
%grund utav att vi måste ta hänsyn till heuristiken. 

%För att veta om vi är färdiga så måste världen uppfylla ett antal logiska
%satser som vi generar utifrån vår tolkning av den meningen som användaren
%anger. Dessa logiska satser skiljer mellan olika parseträd.

%Vi har en timeout på $A^*$ som innebär att vi avbryter om det visar sig att
%algoritmen tar för lång tid. Oftast på grund utav att det inte finns någon
%lösning. 

%Fysiken sköts av en jämnförelseoperator på block. En pyramid eller en boll är alltid
%mindre än allt annat och vi kan bara lägga mindre block på större block. På så
%sätt är det aldrig möjligt att lägga någonting på en boll eller en pyramid.  


%Heuristikfunktionen antar att vi har ett oändligt antal liga platser i världen.
%Dvs vi antar alltid att vi kan flytta ett block inom ramen av två steg. 
%* goal, finished \\


%* heuristic function\\


%* A*, nodes and edges, timeout\\


%* instructions, valid, comparator\\


%* pseudo code A*\\\\

%Successors på en värld returnerar alla möjliga världar man kan komma till genom
%att göra en tillåten instruktion. Add räknar ut f(n) på varje värld
\section{Parser VS. Planner}
The main idea with implementation of of the parser and the planner was that the
parser should take care of the interpretation of a input string and the planner
should take this as a input and perform the actions which should lead to a
solution. However there are cases when the input string may lead to
ambiguous output by the parser. For example the string ''Move all blocks inside a box on top
of the red square.'' makes it possible to move blocks from different boxes. One could
also choose to interpret the string as to move all blocks from every box. Even
further there may be more then one red square in the world. Instead of asking a
attendant questions to the user the idea was the send all possible meanings of
the input string to the planner. The planner should output one and only one plan as simple as
possible from all the possible interpretations. 
\\\\
The parser takes care of a lot of impossible sentences such as ''Take the
floor''. Some sentences will however pass the parser since they don't seem
directly impossible. For example ''Move all blocks inside a box on top of a
square''. This may be impossible due to the laws of physics in the world. Say
for instance that there only is one square in the world and this square is
smaller then some block in a box, then there is no possible solution because it
is impossible to put a bigger block on top of a smaller block. The idea
was here that the planner should realize this and take an other interpretation if
several interpretations where generated by the parser and otherwise fail with
some informative error. For now only a time out messages is  delivered by the
planner.           



