\chapter{Conclusions}
Natural languages are very complex since there are many different
interpretations. We had long discussions about the meaning of ''Above'' and
''Ontop'' and about the difference of these.  
\\\\
During the project we have learned the importance of having a good heuristic
function in A* and the importance of avoiding overestimation.  Several times
when we overestimated the heuristic function our A* algorithm ended up visiting
more than 100 times the numbers of nodes. 
\\\\
We also had a lot of discussions about the difference and how to treat $any$,
$the$ and $all$. We wanted the parser to always give the planner the best parse
tree, which sometimes is impossible because several parse trees could be valid
and similarly. Therefore the parser must sometimes give multiple parse trees to
the planner which must choose the tree that gives the best planning.
\\\\
The planner should always output one and only one plan as simple as
possible from all the possible interpretations. This was harder than we thought. 
There may be a lot of possible interpretation and the format of the
output from the parser became more important. Sadly due to lack of time this
problem became poorly handled and the planner simply takes the first
interpretation delivered by the parser. 
\newpage 

\section{Future work}
Simple future works we would have done if we had more time would be to optimize
the time it take to perform the instructions, restore all blocks that we move
when reaching other blocks or positions. We would also have printed what the
planner does such as ''I move the topmost block from stack X to stack Y''.
\\\\
Also implementing a difference between $any$, $the$ and $all$ in the parser
would be prioritized, together with appropriate changes in the planner.
Furthermore we would have extended the planner such that it do not just take the
first parse tree that is valid. Instead the planner should perform the planning
for the parse tree that got the best planning. 
\\\\
Another interesting work would be to add the possibility of moving a block
in between two other blocks. 
