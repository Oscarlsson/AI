\chapter{Introduction}
Planning is about devising a plan of actions to reach a specific goal
\citep{planning_ai}. There are several different methods that can be used for
the planning part, such as building a search graph or using logic. In this
project, planning was applied using natural language in a dialogue system for a
robot. The task was to, depending on the message given, determine a planning
and perform this planning in the given world. Natural language processing is
used as an important part in, for example, automated online assistants that
provide customer service on a web page. 

\section{Problem description}
The project was about implementing a dialogue system for a robot in a
two-dimensional virtual world, where the robot had an arm to pick up and drop
blocks in the world. The dialogue system was a written dialogue where the user
wrote English sentences which the system translated to a planning and resulting
commands to the robot. The project could be divided into two parts: parse the
given input command and to make and perform an efficient planning. \\\\
The goal was to be able to parse at least four different sentences and to make
and perform an efficient and close to optimal planning for these. 
\\\\
The world contained blocks of different forms and sizes and all moves had to
follow the laws of physics, e.g. no block can be put on a ball and small blocks
cannot hold bigger blocks. \\\\ Another problem that always occurs when working
with natural languages is ambiguity \citep{naturallang_ai}. For example, how to
read "He saw her duck"? This could mean that he saw a duck that belongs to her
or that he saw her move to avoid something. In this project there was a problem
with the input commands which could be ambiguous and therefore result in
several possible plannings. 

\section{Theory}
This project was a subset by a famous PhD thesis called SHRDLU which  is a
dialogue system that operates in a three-dimensional blocks world
\cite{SHRDLU_url}. To parse the sentences, the programming language Grammatical
Framework was used and the planner was done implementing the search algorithm
A*\cite{gf_url,astar_ai}.

\subsection{The blocks world}
The blocks world is a famous planning domain that consists of a set of cubes on
a table \citep{blocksworld_ai}. The blocks can be stacked but only one block
can be on top of another. There is also a robot arm to pick up and move the
blocks. The robot arm can only pick up one block at a time and only the block
at the top of a stack \citep{blocksworld_ai}. The goal is to build one or more
stacks of the blocks, which is specified by which block should be on top of
what other blocks \citep{blocksworld_ai}. An example is shown in Figure
\ref{fig:blocksworld}.
\begin{figure}[h!]
\centering
\includegraphics[scale = 0.4]{fig/blocksworld.png}
\caption{Example of an initial state and goal state in the blocks world \citep{blocksworld_fig}}
\label{fig:blocksworld}
\end{figure}\\
The difference between the blocks world and the blocks world in this project is that the blocks
could have several different forms and sizes, but all moves must follow the
laws of physics. This project operates in a bit more advanced world. 

\subsection{SHRDLU}
SHRDLU is a program, that operates in the blocks world, by Terry Winograd as a
PhD thesis \citep{SHRDLU_url}. SHRDLU is used for understanding natural
language by an interactive English dialog between the computer and a user,
about a small three-dimensional world of blocks \citep{SHRDLU_url}. The program
is able to move various objects in the world, naming collections and answering
general questions about the world \citep{SHRDLU_url2}. SHRDLU also have a
built-in memory and can both remember and discuss its plans and actions
\citep{SHRDLU_url2}.

\newpage
\subsection{Grammatical Framework}
Grammatical Framework, GF, is a programming language for writing grammars of
natural languages. GF is able to parse and generate text in several different
languages using a representation that is language-independent \citep{gf_url}.
GF can generate output in both abstract and concrete syntax and is not
restricted to a specific programming language. It is a functional language but
it is specialized on grammars \citep{gf_url}. 
GF can easily be included in for example Haskell, Java or JavaScript \citep{gf_url}.

\subsection{A*}
A* is a search algorithm which is widely used in path finding and graph
traversal \citep{astar_url}. A* uses best-first search together with Dijkstra's
algorithm and evaluates the nodes according to Equation \ref{Eq:astar}
\citep{astar_url}. \\
\begin{equation}
f(n) = g(n) + h(n)
\label{Eq:astar}
\end{equation}\\
Here $g(n)$ is the cost to reach the node $n$ and $h(n)$ is cost to get from $n$
to the goal \citep{astar_url}. This gives that $f(n)$ is the estimated
cost of the cheapest solution through $n$ \citep{astar_url}. 
\\\\
If the heuristic function $h(n)$ satisfy certain conditions, it could be proven
that A* is both complete and optimal \citep{astar_ai}. For A* to be optimal the
heuristic function must never overestimate the cost to reach the goal
\citep{astar_ai}. This gives that $f(n)$ never overestimates the cost of a
solution either. Also the heuristic function must be consistent, which is
defined as follows: For every node $n$ and every successor $n'$ generated by an
action $a$, the estimated cost for reaching the goal from $n$ cannot be greater
than the step cost of getting to $n'$ plus the estimated cost for reaching the
goal from $n'$ \citep{astar_ai}. This is also described in Equation
\ref{Eq:consistent_astar}.\\
\begin{equation}
h(n) \leq c(n, a, n') + h(n')
\label{Eq:consistent_astar}
\end{equation}\\
It could also be proven that A* is optimally efficient for any consistent
heuristic \citep{astar_ai}. This means that no other optimal algorithm is
guaranteed to expand fewer nodes than A* \citep{astar_ai}. However, A* is not
beneficial for many large-scaled problems since it keeps all generated nodes in
memory so it can run out of space before it runs out of time
\citep{astar_ai}.\\\\ The time complexity of A* depends on the heuristic
function. In the worst case the number of expanded nodes is exponential, but it
is polynomial if the search space is a tree.
