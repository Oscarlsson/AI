\chapter{Results and discussion}
The implemented parser and planner makes a correct parsing and planning for four different example inputs, which are shown in in Table \ref{tab:exampleinput}. As seen at the second example input, it also handles ambiguous inputs and handles multiple parse trees. The first tree both refers to move all blocks in the world and an non-existing block, which is the reason for the error message ''no such block''. 
\begin{table}[h!]
\centering
\begin{tabular}{| p{6cm} | p {5.8cm} | p{1.8cm} | }
\hline
\textbf{Input} & \textbf{Parse trees (concrete syntax)} & \textbf{Planning} \\ \hline
Put the blue block that is to the left of a pyramid in a medium-sized box. & ( move ( the ( thatis ( block \_ \_ blue ) ( leftof ( any ( block pyramid \_ \_ ) ) ) ) ) ( inside ( any ( block box medium \_ ) ) ) ) & pick 0 \linebreak drop 0\linebreak pick 0 \linebreak drop 0\linebreak\\ \hline
Move all wide blocks inside a box on top of the red square. & ( move ( all ( block \_ wide \_ ) ) ( inside ( any ( thatis ( block box \_ \_ ) ( ontop ( the ( block square \_ red ) ) ) ) ) ) ) \newline \newline \newline
( move ( all ( thatis ( block \_ wide \_ ) ( inside ( any ( block box \_ \_ ) ) ) ) ) ( ontop ( the ( block square \_ red ) ) ) ) & "no such block" \newline \newline \newline \newline \newline
pick 0 \linebreak drop 0 \linebreak pick 0 \linebreak drop 0 \linebreak\\ \hline
Put the wide blue block under the black rectangle. & ( move ( the ( block \_ wide blue ) ) ( under ( the ( block rectangle \_ black ) ) ) ) & pick 0 \linebreak drop 0\linebreak \\ \hline
Move all wide rectangles into a red box. & ( move ( all ( block rectangle wide \_ ) ) ( inside ( any ( block box \_ red ) ) ) ) & pick 0 \linebreak drop 0\linebreak\\ \hline
\end{tabular}
\caption{Result of the given example input}
\label{tab:exampleinput}
\end{table}\\
All the examples above refers to a $Move$ action. The implemented parser and planner also handles the actions $Take$ and $Put$. Examples of these are shown below in Table \ref{tab:put_take}.
\begin{table}[h!]
\centering
\begin{tabular}{| p{6cm} | p {5.8cm} | p{1.8cm} | }
\hline
\textbf{Input} & \textbf{Parse trees (concrete syntax)} & \textbf{Planning} \\ \hline
&& \\ \hline
&& \\ \hline
\end{tabular}
\caption{Result of actions $Take$ and $Put$}
\label{tab:put_take}
\end{table}\\
-- några andra meningar vi klarar? T.ex. en take och en put?
\\\\
-- nåt som inte går?

\section{Discussion}
A big problem when handling natural language is that there is many ways of read things which easily makes a sentence ambiguous. We also noticed that it was hard to cover all cases that could occur in the parser and depending on which input we got it was possible to pattern match on power sets of possible inputs, which the code increase rapidly. \\\\
Another difficulty in the beginning was that the syntax tree that was given from the parser was a concrete syntax tree and we wanted an abstract syntax tree to be able to pattern match on the different parts. This was solved when we find a library which could convert between GF grammar as concrete and abstract syntax. \\\\
We also choosed to implement a planning algorithm by graph traversal, by implementing A*. Another possibility would been to use for example First-Order Logic instead. 
\\\\
-- skriv nåt om heuristikfunktionen

